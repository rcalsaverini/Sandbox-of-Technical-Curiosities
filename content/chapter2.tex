\chapter{Combinatorics and Derivatives: Faà di Bruno's Formula}

\section{Introduction: The Explosion of the Chain Rule}

\lettrine{T}{he} Chain Rule is one of the first distinct "rules" of calculus we learn that feels like a machine. It tells us that to differentiate a composite function $h(x) = F(G(x))$, we peel it like an onion: differentiate the outside, then multiply by the derivative of the inside.
\begin{equation}
    h'(x) = F'(G(x)) \cdot G'(x)
\end{equation}
For a single derivative, this is elegant and manageable. However, in advanced physics—whether we are calculating high-order corrections in perturbation theory, expanding free energy in statistical mechanics, or evaluating Feynman diagrams—we rarely stop at the first derivative. We need the $n$-th derivative.

If we attempt to calculate these higher derivatives by brute force, applying the product and chain rules iteratively, the expression explodes in complexity almost immediately.

\subsection{The Brute Force Expansion}
Let us calculate the first three derivatives explicitly to see the structure emerge. We suppress the arguments for brevity, letting $F \equiv F(G(x))$ and $G \equiv G(x)$.

\textbf{First Derivative ($n=1$):}
\begin{equation}
    h' = F' G'
\end{equation}

\textbf{Second Derivative ($n=2$):}
Applying the product rule to the result of $n=1$:
\begin{equation}
    \begin{split}
        h'' &= \dv{x} (F' G') \\
            &= (\dv{x} F') G' + F' (\dv{x} G') \\
            &= (F'' G') G' + F' G'' \\
            &= F'' (G')^2 + F' G''
    \end{split}
\end{equation}

\textbf{Third Derivative ($n=3$):}
Differentiating again requires applying the product rule to two distinct terms:
\begin{equation}
    \begin{split}
        h''' &= \dv{x} \qty[ F'' (G')^2 + F' G'' ] \\
             &= \underbrace{\qty[ F''' G' \cdot (G')^2 + F'' \cdot 2G' G'' ]}_{\text{Deriving } F'' (G')^2} + \underbrace{\qty[ F'' G' \cdot G'' + F' G''' ]}_{\text{Deriving } F' G''} \\
             &= F''' (G')^3 + 2F'' G' G'' + F'' G' G'' + F' G''' \\
             &= F''' (G')^3 + 3F'' G' G'' + F' G'''
    \end{split}
\end{equation}

\subsection{Pattern Recognition}
Already at $n=3$, we can observe a non-trivial structure. The derivative is a sum of terms where:
\begin{enumerate}
    \item The total order of differentiation on the inner function $G$ always sums to $n$ (e.g., in $G' G''$, $1+2=3$).
    \item The order of the outer derivative $F^{(k)}$ corresponds to the number of factors of $G$ present.
    \item There are integer coefficients (like the \textbf{3} in $3F'' G' G''$) that are not immediately obvious from the functions themselves.
\end{enumerate}

To understand the origin of these coefficients, it helps to visualize the derivative not as an equation, but as a set of "trees". Each term represents a specific way to group the differentiations.

\begin{figure}[H]
    \centering
    \begin{tikzpicture}[
        node distance=1.5cm,
        level 1/.style={sibling distance=1.5cm},
        level 2/.style={sibling distance=1cm},
        fnode/.style={circle, draw=black, fill=gray!10, thick, minimum size=8mm},
        gnode/.style={circle, draw=black, thick, minimum size=6mm},
        leaf/.style={font=\footnotesize}
    ]

    % Term 1: F' G'''
    \begin{scope}[xshift=-4cm]
        \node[fnode] (F) {$F'$}
            child {node[gnode] (G) {$G$}
                child {node[leaf] {$\dv{x}$}}
                child {node[leaf] {$\dv{x}$}}
                child {node[leaf] {$\dv{x}$}}
            };
        \node[below=3cm of F] (T) {Term: $F' G'''$};
        \node[below=0.2cm of T, font=\small\itshape] 
            {$\substack{\{1,2,3\}}$};
    \end{scope}

    % Term 2: F'' G' G''
    \begin{scope}[xshift=0cm]
        \node[fnode] (F) {$F''$}
            child {node[gnode] (G1) {$G$}
                child {node[leaf] {$\dv{x}$}}
            }
            child {node[gnode] (G2) {$G$}
                child {node[leaf] {$\dv{x}$}}
                child {node[leaf] {$\dv{x}$}}
            };
        \node[below=3cm of F] (T) {Term: $3 F'' G' G''$};
        \node[below=0.2cm of T, font=\small\itshape] 
            {$\substack{\{1\}\{2,3\} \\ \{2\}\{1,3\} \\ \{3\}\{1, 2\}}$};
    \end{scope}

    % Term 3: F''' (G')^3
    \begin{scope}[xshift=4cm]
        \node[fnode] (F) {$F'''$}
            child {node[gnode] {$G$} child {node[leaf] {$\dv{x}$}}}
            child {node[gnode] {$G$} child {node[leaf] {$\dv{x}$}}}
            child {node[gnode] {$G$} child {node[leaf] {$\dv{x}$}}};
        \node[below=3cm of F] (T) {Term: $F''' (G')^3$};
        \node[below=0.2cm of T, font=\small\itshape] 
            {$\substack{\{1\}\{2\}\{3\}}$};            
    \end{scope}

    \end{tikzpicture}
    \caption{Visualizing the third derivative. The structure of the terms corresponds to the different ways we can partition the three derivatives (leaves) among the inner functions $G$.}
    \label{fig:derivative_trees}
\end{figure}

This visualization suggests that the coefficients are combinatorial in nature. They count the number of specific ways we can ``distribute'' $n$ differentiation events among the internal functions. To understand this for general $n$, we must move away from algebraic brute force and adopt a combinatorial perspective.

\section{The Univariate Case: $D^n [F(G(x))]$}

To systematically derive the formula for the $n$-th derivative, we employ a standard combinatorial trick: we temporarily assume that every differentiation is performed with respect to a \textit{distinct} variable. Instead of computing $\dv[n]{}{x}$, we compute the mixed partial derivative with respect to distinct variables $x_1, x_2, \dots, x_n$, and set them all equal to $x$ at the very end.
\begin{equation}
    \dv[n]{}{x} F(G(x)) = \eval{\pdv{}{x_1} \pdv{}{x_2} \dots \pdv{}{x_n} F(G(x))}_{x_1 = \dots = x_n = x}
\end{equation}
This change of perspective is powerful because it allows us to track exactly which differentiation operator is responsible for which term.

\subsection{The Concept of Set Partitions}
This approach naturally leads us to the concept of a \textbf{Set Partition}. A partition $\pi$ of a set $S = \{1, \dots, n\}$ is a collection of disjoint non-empty subsets (called \textit{blocks}) $B_1, \dots, B_k$ whose union is $S$.

For example, for $n=3$ (corresponding to the third derivative), the set $\{1, 2, 3\}$ has 5 possible partitions:
\begin{itemize}
    \item \textbf{1 block:} $\{\{1, 2, 3\}\}$
    \item \textbf{2 blocks:} $\{\{1, 2\}, \{3\}\}, \{\{1, 3\}, \{2\}\}, \{\{1\}, \{2, 3\}\}$
    \item \textbf{3 blocks:} $\{\{1\}, \{2\}, \{3\}\}$
\end{itemize}
We denote the set of all partitions of $\{1, \dots, n\}$ as $\Pi_n$. For a partition $\pi \in \Pi_n$, we denote the number of blocks as $|\pi|$ (or sometimes $k$) and the size of a specific block $B$ as $|B|$.

\subsection{Combinatorial Derivation}
Let us view differentiation as an iterative process. Let $G_B$ denote the derivative of $G$ with respect to the variables in the set $B$. For example, $G_{\{1,3\}} = \pdv{}{x_1} \pdv{}{x_3} G$.

Consider the result for $n=2$ with distinct variables:
\begin{equation}
    \pdv{}{x_1}\pdv{}{x_2} F(G) = F'' \cdot G_{\{1\}} G_{\{2\}} + F' \cdot G_{\{1,2\}}
\end{equation}
Now, apply the next derivative operator $\pdv{}{x_3}$ to this expression. By the product rule, $\pdv{}{x_3}$ must act on one of the existing factors in each term.

\begin{figure}[H]
    \centering
    \begin{tikzpicture}[
        node distance=1cm and 1.5cm,
        every node/.style={font=\small},
        term/.style={rectangle, draw=black!60, thick, rounded corners, align=center, minimum width=2.5cm, fill=white},
        arrow/.style={->, >=stealth, thick, color=black!70},
        label/.style={fill=white, inner sep=2pt, font=\footnotesize, text=gray!80!black}
    ]

    % State n=2
    \node[term] (Term1) {Current Term ($n=2$):\\$F' \cdot G_{\{1,2\}}$\\Partition: $\{\{1,2\}\}$};
    
    % Children for n=3
    \node[term, below left=of Term1] (Child1) {\textbf{Option A: Hit Outer $F$}\\$F'' \cdot G_{\{1,2\}} G_{\{3\}}$\\Partition: $\{\{1,2\}, \{3\}\}$};
    \node[term, below right=of Term1] (Child2) {\textbf{Option B: Hit Inner $G$}\\$F' \cdot G_{\{1,2,3\}}$\\Partition: $\{\{1,2,3\}\}$};

    % Arrows
    \draw[arrow] (Term1) -- node[label, left, xshift=-0.2cm] {New block $\{3\}$} (Child1);
    \draw[arrow] (Term1) -- node[label, right, xshift=0.2cm] {Join block} (Child2);

    \end{tikzpicture}
    \caption{The differentiation process as partition building. Applying $\pdv{}{x_3}$ to an existing term corresponds to either creating a new block (differentiating $F$) or adding the element 3 to an existing block (differentiating $G$).}
    \label{fig:partition_process}
\end{figure}

As shown in Figure \ref{fig:partition_process}, applying a new derivative $\pdv{}{x_{n+1}}$ to an existing term corresponding to a partition $\pi$ results in two possibilities:
\begin{enumerate}
    \item \textbf{Hit the Outer Function $F^{(k)}$:} This increases the outer derivative order to $F^{(k+1)}$ and brings down a new factor $G_{\{n+1\}}$ by the chain rule. Combinatorially, this adds the singleton block $\{n+1\}$ to the partition.
    \item \textbf{Hit an Inner Function $G_B$:} This turns $G_B$ into $G_{B \cup \{n+1\}}$. Combinatorially, this inserts the element $n+1$ into the existing block $B$.
\end{enumerate}

This recursive structure covers every possible way to form a partition. Therefore, the $n$-th derivative is simply the sum over all possible partitions of the set $\{1, \dots, n\}$.

\subsection{The Faà di Bruno Formula}
Summing over all partitions $\pi \in \Pi_n$, we arrive at the general formula. For each partition, the order of the outer derivative is the number of blocks $|\pi|$, and the inner derivatives correspond to the sizes of the blocks $|B|$.

\begin{equation} \label{eq:faa_partition}
    \dv[n]{}{x} F(G(x)) = \sum_{\pi \in \Pi_n} F^{(|\pi|)}(G(x)) \cdot \prod_{B \in \pi} G^{(|B|)}(x)
\end{equation}

While elegant, the sum over set partitions can be computationally redundant because many partitions yield the same numerical term (e.g., $\{\{1,2\}, \{3\}\}$ and $\{\{1,3\}, \{2\}\}$ both result in $F'' G'' G'$). In practice, we often group these terms by the \textit{shape} of the partition (integer partitions), leading to the Bell Polynomial form, which we will discuss in later sections.

\section{The Mixed Case: $D^n [F(G(x), x)]$}

In many physical applications, such as perturbation theory or Lagrangian mechanics, a variable $x$ often influences a function $F$ through two distinct channels: implicitly through an intermediate function $G(x)$ and explicitly as a direct argument. We consider the composite function:
\begin{equation}
    k(x) = F(G(x), x)
\end{equation}
Here, we distinguish the arguments by writing $F(u, v)$ where $u = G(x)$ and $v = x$.

\subsection{The Two Routes of Change}
Differentiation with respect to $x$ now splits into two paths. The Chain Rule tells us that the total derivative operator $D_x$ is the sum of two partial operators:
\begin{equation}
    \dv{x} = \underbrace{\pdv{G}{x} \pdv{}{u}}_{\text{Composite Path}} + \underbrace{\pdv{}{v}}_{\text{Direct Path}}
\end{equation}
The "Composite Path" involves the chain rule through $G$, while the "Direct Path" is a simple partial derivative.

\begin{figure}[H]
    \centering
    \begin{tikzpicture}[
        node distance=2cm,
        every node/.style={font=\small},
        var/.style={circle, draw=black, thick, minimum size=8mm, fill=white},
        func/.style={rectangle, draw=black!60, thick, rounded corners, minimum width=1.5cm, fill=gray!10},
        arrow/.style={->, >=stealth, thick, color=black!70}
    ]

    \node[var] (x) {$x$};
    \node[func, above left=1.5cm and 1cm of x] (u) {$u=G(x)$};
    \node[var, above right=1.5cm and 1cm of x] (v) {$v=x$};
    \node[func, above=1.5cm of u] (F) at ($(u)!0.5!(v) + (0, 1.5cm)$) {$F(u,v)$};

    \draw[arrow] (x) -- node[left, font=\footnotesize] {Composite} (u);
    \draw[arrow] (x) -- node[right, font=\footnotesize] {Direct} (v);
    \draw[arrow] (u) -- (F);
    \draw[arrow] (v) -- (F);

    \end{tikzpicture}
    \caption{The dependency graph for the mixed case. The variable $x$ influences $F$ through two routes. Higher derivatives involve distributing differentiation operators across these two paths.}
    \label{fig:dependency_dag}
\end{figure}

\subsection{The Leibniz-Faà Synthesis}
To find the $n$-th derivative $\dv[n]{k}{x}$, we treat the total derivative as the sum of two commuting operators: $D_x = \mathcal{D}_{comp} + \mathcal{D}_{dir}$. Since these operators act on different slots of $F$ (and assuming $F$ is smooth enough that mixed partials commute), we can apply the \textbf{Binomial Theorem} directly to the operator itself:
\begin{equation}
    D_x^n = (\mathcal{D}_{comp} + \mathcal{D}_{dir})^n = \sum_{j=0}^n \binom{n}{j} \mathcal{D}_{dir}^j \mathcal{D}_{comp}^{n-j}
\end{equation}

\subsection{Combinatorial Interpretation}
This formula has a beautiful combinatorial interpretation involving two stages of choice:
\begin{enumerate}
    \item \textbf{Choosing the Path:} The binomial coefficient $\binom{n}{j}$ counts the ways to choose which $j$ of the $n$ differentiation indices take the "Direct Path". These become simple partial derivatives $\pdv[j]{}{v}$.
    \item \textbf{Partitioning the Rest:} The remaining $n-j$ indices take the "Composite Path". As we established in Section 2, differentiating through a composite function requires summing over all set partitions. Thus, we apply the univariate Faà di Bruno formula to these $n-j$ derivatives.
\end{enumerate}

\subsection{The Master Formula}
Combining the Binomial expansion for the two paths with the Faà di Bruno expansion for the composite path yields the general formula for the mixed case:

\begin{equation}
    \dv[n]{k}{x} = \sum_{j=0}^n \binom{n}{j} \sum_{\pi \in \Pi_{n-j}} \underbrace{\frac{\partial^{j + |\pi|} F}{\partial v^j \partial u^{|\pi|}}}_{\text{Mixed Partials of } F} \cdot \underbrace{\prod_{B \in \pi} G^{(|B|)}(x)}_{\text{Partitions of } G}
\end{equation}

This formula is the engine behind high-order perturbation theory. It allows us to systematically disentangle the effects of the parameter $u$ (Direct Path) from the effects of the solution trajectory $x(u)$ (Composite Path).

\section{Computational Tools: Bell Polynomials}

While the summation over set partitions (Equation \ref{eq:faa_partition}) provides a clear conceptual picture of the derivative, it is often redundant for actual calculation. 

Consider the $n=3$ case. We identified five possible set partitions:
\[
\{\{1,2,3\}\}, \quad \{\{1,2\},\{3\}\}, \quad \{\{1,3\},\{2\}\}, \quad \{\{2,3\},\{1\}\}, \quad \{\{1\},\{2\},\{3\}\}
\]
Notice that the three partitions in the middle—$\{\{1,2\},\{3\}\}$ and its permutations—all describe the same physical situation: the outer function $F$ is differentiated twice, one inner $G$ is differentiated twice, and one inner $G$ is differentiated once. All three terms evaluate to:
\[
    F'' \cdot G'' \cdot G'
\]
Since the variables $x_1, x_2, x_3$ are eventually all set to $x$, these terms are numerically identical. It is much more efficient to group these terms by the \textit{shape} of the partition, known as the \textbf{Integer Partition}.

\begin{figure}[H]
    \centering
    \begin{tikzpicture}[
        node distance=1.5cm,
        setpart/.style={rectangle, draw=blue!50, fill=blue!5, thin, rounded corners, font=\footnotesize, align=center, minimum width=2cm},
        intpart/.style={rectangle, draw=red!50, fill=red!5, thick, rounded corners, font=\small, align=center, minimum width=2.5cm},
        arrow/.style={->, >=stealth, thick, color=gray!50}
    ]

    % Integer Partitions (Right)
    \node[intpart] (I1) {Shape: $3$\\ (1 block of size 3)};
    \node[intpart, below=1.5cm of I1] (I2) {Shape: $2+1$\\ (1 size 2, 1 size 1)};
    \node[intpart, below=1.5cm of I2] (I3) {Shape: $1+1+1$\\ (3 blocks of size 1)};

    % Set Partitions (Left)
    \node[setpart, left=3cm of I1] (S1) {$\{\{1,2,3\}\}$};
    
    \node[setpart, left=3cm of I2] (S2b) {$\{\{1,3\},\{2\}\}$};
    \node[setpart, above=0.2cm of S2b] (S2a) {$\{\{1,2\},\{3\}\}$};
    \node[setpart, below=0.2cm of S2b] (S2c) {$\{\{2,3\},\{1\}\}$};

    \node[setpart, left=3cm of I3] (S3) {$\{\{1\},\{2\},\{3\}\}$};

    % Mappings
    \draw[arrow] (S1) -- (I1);
    \draw[arrow] (S2a.east) -- (I2.west);
    \draw[arrow] (S2b.east) -- (I2.west);
    \draw[arrow] (S2c.east) -- (I2.west);
    \draw[arrow] (S3) -- (I3);

    % Label
    \node[draw=none, above=0.5cm of S2a, color=blue!70!black] {\textbf{Set Partitions} ($\Pi_n$)};
    \node[draw=none, above=0.5cm of I1, color=red!70!black] {\textbf{Integer Partitions}};

    \end{tikzpicture}
    \caption{Collapsing Set Partitions into Integer Partitions. The multiplicity of the integer partition corresponds to the number of ways to form that specific shape from distinct elements.}
    \label{fig:partitions_collapse}
\end{figure}

\subsection{Partial Bell Polynomials}
The coefficients arising from this grouping are encoded in the \textbf{Partial Bell Polynomials}, denoted $B_{n,k}$.

Before introducing the algebraic formula involving factorials, it is insightful to define the Bell polynomial via its combinatorial meaning. $B_{n,k}(x_1, \dots, x_{n-k+1})$ is simply the sum of products of variables $x_j$ over all possible set partitions of $\{1, \dots, n\}$ into exactly $k$ blocks:
\begin{equation}
    B_{n,k}(x_1, \ldots, x_{n-k+1}) = \sum_{\pi \in \Pi_{n,k}} \prod_{B\in\pi} x_{|B|}
\end{equation}
where $\Pi_{n,k}$ is the set of all ways of partitioning $\{1, \ldots, n\}$ into exactly $k$ non-empty subsets. This definition confirms that the Bell polynomial acts as a "collector" for all Faà di Bruno terms corresponding to a specific outer derivative order $k$.

To compute this efficiently without enumerating partitions, we use the algebraic formula that accounts for the combinatorial multiplicities (the shapes):
\begin{equation}
    B_{n,k}(x_1, \dots, x_{n-k+1}) = \sum \frac{n!}{j_1! j_2! \dots j_{n-k+1}!} \prod_{m=1}^{n-k+1} \left( \frac{x_m}{m!} \right)^{j_m}
\end{equation}
where the sum is over all non-negative integers $j_m$ (representing the number of blocks of size $m$) satisfying two constraints:
\begin{enumerate}
    \item Total number of blocks: $\sum_{m} j_m = k$
    \item Total number of elements: $\sum_{m} m \cdot j_m = n$
\end{enumerate}

\subsection{The Compact Formula}
Using these polynomials, we can rewrite the Faà di Bruno formula in a form that is computationally ready for implementation in computer algebra systems:

\begin{equation}
    \dv[n]{}{x} F(G(x)) = \sum_{k=1}^n F^{(k)}(G(x)) \cdot B_{n,k}\left( G'(x), G''(x), \dots, G^{(n-k+1)}(x) \right)
\end{equation}

Furthermore, for the \textbf{Mixed Case} discussed in Section 3 (relevant to perturbation theory), we can substitute the Bell polynomial back into our master equation:

\begin{equation}
    \dv[n]{k}{x} = \sum_{j=0}^n \binom{n}{j} \sum_{i=0}^{n-j} \underbrace{\frac{\partial^{j+i} F}{\partial v^j \partial u^i}}_{\text{Mixed Partials}} \cdot B_{n-j, i}\left( G', G'', \dots \right)
\end{equation}

This is the explicit formula used to generate the recursive equations for the perturbation coefficients $x_k$ derived in the previous chapter.
