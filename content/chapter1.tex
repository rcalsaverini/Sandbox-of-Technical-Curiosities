\chapter{Differential Equations}

\section{The 1D Schrödinger Equation as a Non-Linear System}

\lettrine{T}{he 1D Schrödinger Equation} is typically celebrated for its linearity, a property that allows for the superposition of states.
\begin{equation}
    \label{eq:schrodinger-original}
    -\frac{\hbar^2}{2m}\dv[2]{\psi}{x} + V(x) \psi = E \psi
\end{equation}
Rearranging this equation a bit, we get to a more stripped down version:
\begin{eqnarray*}
    -\frac{\hbar^2}{2m}\dv[2]{\psi}{x} + (V(x) - E) \psi &= 0 \\
    \frac{\hbar^2}{2m}\dv[2]{\psi}{x} + (E - V(x)) \psi &= 0
\end{eqnarray*}
Finally, defining $u(x) = \frac{2m}{\hbar^2}(E - V(x))$, we get:
\begin{equation}
    \label{eq:schrodinger}
    \dv[2]{\psi}{x} + u(x) \psi = 0
\end{equation}
which looks a bit like a harmonic oscillator with a "time-dependent frequency" $\sqrt{u(x)}$.

\subsection{The Exponential Ansatz}
What happens if we force a non-linear perspective? Let us propose the ansatz:
\begin{equation}
    \psi(x) = \exp{h(x)}
\end{equation}
We now calculate the derivatives to substitute back into \eqref{eq:schrodinger}. The first derivative is straightforward by the chain rule:
\begin{equation}
    \dv{x} e^{h(x)} = h'(x) \psi(x)
\end{equation}
The second derivative requires the product rule. This is where the non-linearity emerges:
\begin{align}
    \dv[2]{x} e^{h(x)} & = \dv{x} \left( h'(x) \psi(x) \right) \nonumber                 \\
                       & = h''(x) \psi(x) + h'(x) \dv{\psi(x)}{x} \nonumber              \\
                       & = h''(x) \psi(x) + h'(x) \left( h'(x) \psi(x) \right) \nonumber \\
                       & = \left[ h''(x) + \left( h'(x) \right)^2 \right] \psi(x)
\end{align}

\subsection{The Resulting Equation}
Substituting this back into the original equation \eqref{eq:schrodinger}, we can factor out the wavefunction $\psi(x)$ (which is non-zero by definition of the exponential).

\begin{result}{The Modified Riccati Form}{riccati_form}
    The linear Schrödinger equation is equivalent to the following non-linear differential equation for the exponent $h(x)$:
    \begin{equation}
        \boxed{ h''(x) + \left( h'(x) \right)^2 = -u(x) }
    \end{equation}
\end{result}

This turned our non-homogeneous linear "harmonic oscillator" into a homogeneous forced system, but with a non-linear term.

\begin{curiosity}{The Riccati Connection}{ricatti_connection}
    If we define a new variable $v(x) = h'(x)$ (the logarithmic derivative of the wavefunction), the equation above becomes:
    \[
        v'(x) + v(x)^2 + u(x) = 0
    \]
    This is the famous \textbf{Riccati Equation}. It shows a deep duality: a second-order linear ODE can always be reduced to a first-order non-linear ODE. We traded order for linearity.

\end{curiosity}